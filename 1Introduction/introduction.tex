\chapter{INTRODUCTION}
% (20% of Proposal Length)
\pagenumbering{arabic}



\section{Introduction}
"Mitho Delivery" is an online food delivery platform that brings together customers and local restaurants for a hassle-free dining experience. This user-friendly webpage allows customers to explore restaurant menus and place orders effortlessly. With a strong commitment to quality and customer satisfaction, Mitho Delivery guarantees timely delivery of delicious meals to customers' doorsteps. As the online food delivery trend continues to reshape dining habits, our project, "Mitho Delivery: Joy in Food," focuses on creating an intuitive and responsive website. This platform aims to connect users with a diverse array of restaurants and culinary delights, enhancing the joy of relishing favorite dishes from the comfort of their homes or workplaces.

\section{Problem Statement}
Online food delivering systems may encounter various challenges that can negatively impact the customer experience. One common issue is late delivery, which can occur due to logistical or operational issues. Technical issues with the website can also disrupt the ordering process, causing frustration for customers. Another problem is that the limited coverage area of the delivery service, which can restrict access for customers in certain regions. Some platforms may impose additional service charges, leading to dissatisfaction among customers.\\\\ Furthermore, if employee services are not well-managed, it can result in delays and poor service quality. Lastly, a lack of care for the products ordered by customers, such as mishandling or inadequate packaging, can lead to disappointment and dissatisfaction. Addressing these challenges is crucial for online food delivering systems to maintain customer satisfaction and loyalty.

\section{Objectives}
\begin{itemize}
    \item To provide customers with a convenient and hassle-free way to order local food from a wide range of local restaurants, saving them time and effort.
    \item Develop an intuitive and user-friendly online food ordering platform for both users and restaurants.
\end{itemize}

\section{Scope and Limitation}
\subsection{Scope}
This project aims to benefit small food businesses by providing them with a stable economic flow. By offering an online food delivery platform, these businesses can expand their customer reach and increase their sales, leading to a more sustainable revenue stream. Additionally, this project strives to enhance the lives of middle-class families in Nepal by offering them convenient access to a wide range of local food options from the comfort of their homes. By expanding the services across Nepal, this project aims to reach a larger population and ensure that more people can benefit from the convenience of online food delivery. By streamlining operations and providing better employee facilities, this project can help reduce extra service charges and enhance the overall experience for both customers and employees. Ultimately, the systematic approach of this project aims to satisfy customers' and restaurants needs effectively and efficiently.

\subsection{limitation}
\begin{enumerate}
\item Our system have limited areas particularly in remote areas.
\item Doesn’t support offline mode and requires internet.
\item Limited menu, available food options for delivery may be fewer in number compared to what is offered at a physical restaurant
\end{enumerate}

\section{Potential Applications}
The potential applications of this project are far-reaching and hold value across various domains:
\begin{itemize}
    \item Small Food Businesses: The primary application lies in supporting small local food businesses, offering them a digital avenue to expand their customer base and increase revenue. This project can empower local eateries, cafes, and food vendors, providing them with a platform to reach a wider audience and thrive in a competitive market.
    \item Tourism and Travel: Tourists and travelers often explore local cuisine when visiting new places. This project can serve as a guide, allowing visitors to sample authentic local foods of that area without leaving their accommodation, enhancing their travel experience.
    \item Flexible Employment: The expansion of online food delivery creates opportunities for individuals seeking flexible employment, such as part-time delivery drivers. This can serve as an additional income source for those looking for flexible work arrangements.
    \item Convenience for Special Needs: Online food delivery platforms can provide a convenient option for individuals with special needs or limited mobility, ensuring that they can enjoy a variety of meals without leaving their homes.
\end{itemize}

\section{Orginality of Project}
\begin{itemize}
    \item Inclusive Restaurant Network: An inclusive restaurant network is being introduced, encompassing not only established eateries but also previously undiscovered culinary treasures that might not have been featured online previously. This approach broadens the array of choices accessible to customers and enhances the vibrancy of the local food scene.
    \item Localized Cultural Insight: Understanding the cultural differences of a region is essential in the food industry. Mitho Delivery places emphasis on showcasing local delicacies, culinary traditions, and dishes specific to a particular area, thereby preserving and sharing cultural heritage.
    \item Enhanced Customization: Unlike many platforms, Mitho Delivery focuses on enhancing customization options for users. This includes personalized dish modifications hereby offering a more individualized dining experience.
\end{itemize}

\newpage
\section{Report Organisation}
The material in this project report is organised into six chapters. After this introductory chapter introduces the problem topic this research tries to address, chapter 2 contains the literature review of vital and relevant publications, pointing toward a notable research gap. Chapter 3 describes system design and analysis for the of this project. Chapter 4 provides an overview of what tools have been used and testing of this project .chapter 5 discusses the project .Chapter 6 conludes the project.